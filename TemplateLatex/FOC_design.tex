\section{Design of FOC controllers} 

\label{sec:FOC}

This section focuses on the design of the Field-Oriented Control (FOC) system for the Synchronous Reluctance Motor (SyRM), relying on previously extracted motor parameters.

\subsection{PI current controller tuning}

The PI controllers for the $d$- and $q$-axis current loops are designed using a standard method based on the motor time constants:

\begin{equation}
\tau_d = \frac{L_d}{R_s}, \quad \tau_q = \frac{L_q}{R_s}
\end{equation}

Given:
\begin{itemize}
    \item $R_s = 0.127~\Omega$
    \item $L_d = \texttt{TBD}~\text{H}$ \hfill \textit{\footnotesize (apparent inductance from flux map)}
    \item $L_q = \texttt{TBD}~\text{H}$
\end{itemize}

Assuming a desired current control bandwidth $\omega_b = \dfrac{10}{\tau}$, the proportional and integral gains are:

\begin{align}
k_{p,d} &= \omega_{b,d} \cdot L_d = \texttt{TBD} \\
k_{i,d} &= k_{p,d} \cdot \frac{\omega_{b,d}}{5} = \texttt{TBD} \\
k_{p,q} &= \omega_{b,q} \cdot L_q = \texttt{TBD} \\
k_{i,q} &= k_{p,q} \cdot \frac{\omega_{b,q}}{5} = \texttt{TBD}
\end{align}

These gains will be refined after initial simulations using MATLAB/Simulink.

\subsection{Speed controller design}

The speed loop typically uses a PI controller tuned based on the desired closed-loop bandwidth and inertia. As the mechanical parameters are not provided in this phase, this section will be completed after further data is available.

\vspace{2cm} %remove it from final project
